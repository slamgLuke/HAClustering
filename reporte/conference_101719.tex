\documentclass[conference]{IEEEtran}
\IEEEoverridecommandlockouts

\usepackage{cite}
\usepackage{amsmath,amssymb,amsfonts}
\usepackage{algorithmic}
\usepackage{graphicx}
\usepackage{textcomp}
\usepackage{xcolor}
\usepackage{hyperref}
\usepackage{placeins}

\def\BibTeX{{\rm B\kern-.05em{\sc i\kern-.025em b}\kern-.08em T\kern-.1667em\lower.7ex\hbox{E}\kern-.125emX}}

\begin{document}
    % Title
    \title{Human Activity Recognition using Clustering\\}

    % Authors
    \author{\textit{Lucas Carranza, Guillermo Sánchez, David Herencia, José Osnayo}}
\maketitle

\section{Introducción}
En el presente documento se 

\section{Conjunto de Datos}
\subsection{Exploración del Dataset}
El conjunto de datos utilizado es

\subsection{Análisis de Frecuencias}

Analizamos las frecuencias de cada clase utilizando la librería \textit{Seaborn} para graficar el histograma.
\begin{comment}
\begin{figure}
    \centering
    \includegraphics[width=\linewidth]{class_freq.png}
    \caption{Histograma de Frecuencias de Clases}
    \label{fig1}
\end{figure}
\end{comment}

Observamos que...

\subsection{Extracción de Features}

\section{Metodología}
% Explanation of the model, loss functions, and regularization techniques

\subsection{Modelo 1}


\subsection{Modelo 2}


\section{Implementación}
% Include the link to Colab or GitHub where the implementation can be found, avoiding direct
% code placement in the report. Define a seed to replicate the results. [Optional] Relevant implementation
% details can also be included (error handling, parallelization, etc.).
El código documentado del proyecto se encuentra en el siguiente 
\href{https://github.com/slamgLuke/HAClustering}{Repositorio de Github}

\noindent \textbf{Detalles de implementación:}
\begin{itemize}
\item Se fijó la semilla aleatoria para permitir la replicación de resultados.
\end{itemize}


\section{Experimentación}
% Present results with graphs and/or tables, avoiding terminal screenshots
Se experimentaron principalmente con los dos modelos mencionados. Se utilizó K-Fold Cross-Validation con

\subsection{Experimentación con modelo1}
\subsection{Experimentación con modelo2}

\section{Discusión}
% Interpretation of the obtained results and their relationship with the learned theory
Tras haber realizado la experimentación, discutiremos los resultados obtenidos.

\begin{itemize}
\item 
\end{itemize}

\section{Conclusiones}
% Summary of results, limitations, and recommendations

\begin{enumerate}
\item 
\end{enumerate}

\end{document}
